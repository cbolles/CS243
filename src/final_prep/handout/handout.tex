\documentclass[]{article}

\usepackage[margin=0.75in]{geometry}
\usepackage{listings}
\usepackage{xcolor}
\usepackage [english]{babel}
\usepackage [autostyle, english = american]{csquotes}
\MakeOuterQuote{"}

\definecolor{codegreen}{rgb}{0,0.6,0}
\definecolor{codegray}{rgb}{0.5,0.5,0.5}
\definecolor{codepurple}{rgb}{0.58,0,0.82}
\definecolor{backcolour}{rgb}{0.95,0.95,0.92}

\lstdefinestyle{mystyle}{
    backgroundcolor=\color{backcolour},
    commentstyle=\color{codegreen},
    keywordstyle=\color{magenta},
    numberstyle=\tiny\color{codegray},
    stringstyle=\color{codepurple},
    basicstyle=\ttfamily\footnotesize,
    breakatwhitespace=false,
    breaklines=true,
    captionpos=b,
    keepspaces=true,
    numbers=left,
    numbersep=5pt,
    showspaces=false,
    showstringspaces=false,
    showtabs=false,
    tabsize=2
}

\lstset{style=mystyle}

\begin{document}

\section*{Activity 1 - Program Translation}
Organize the following steps of program translation into the order that they occure. Linking,
Compiling, Preprocessing, Assembling. Additionally, where does lexical, syntax, and semantic
analysis take place? What are errors that can take place during each step of program translation?

\section*{Activity 2 - Bitwise Operators}
Your boss assigns you the task of reading in data from an accelerometer and pulling out the useful
information. You look at the data sheet and see that each time you read from the accelerometer you
get back an integer. Inside the integer is packed the status of the chip, the x acceleration, y
acceleration, and z acceleration in that order. Write functions to pull out the information to 
match the code snippet below using the fact that the largest a byte can hold is 255.
\begin{lstlisting}[language=C]
int main() {
      uint32_t raw_value = read_acc(); // Assume this returns the value from the accelerometer
      uint8_t status = get_status(raw_value);
      uint8_t x_acc = get_x_acc(raw_value);
      uint8_t y_acc = get_y_acc(raw_value);
      uint8_t z_acc = get_z_acc(raw_value);

      printf("The original value was %#010X\n", raw_value);
      printf("The status was %#4X\n", status);
      printf("The x_acc was %#4X\n", x_acc);
      printf("The y_acc was %#4X\n", y_acc);
      printf("The z_acc was %#4X\n", z_acc);

      return 0;
}
\end{lstlisting}

\section*{Activity 3 - System and Process Level IO}
Fill in the table comparing system level IO to the process level IO that we use when writting code.

\begin{centering}
    \begin{tabular}{|c|p{4cm}|p{5cm}|}
        \hline
        & Our Program & OS \\
        \hline
        File Representation & & \\
        \hline
        Opening a File & & \\
        \hline
        Reading in byte oriented data & & \\
        \hline
        Return value when reading/writing byte oriented data & & \\
        \hline
        Uses buffereing to reduce calls to lower level & & \\
        \hline
        Close a file & & \\
        \hline
        Manipulate offset in file & & \\
        \hline
        Flush buffer & & \\
        \hline
    \end{tabular}
\end{centering}

\section*{Activity 4 - Unions}
What is the output of the code snippet below.

\begin{lstlisting}
#include <stdio.h>

union ex {
    int num;
    char c;
};

int main() {
    union ex my_union;
    my_union.num = 65;
    printf("The size of union ex is %lu\n", sizeof(union ex));
    printf("The char in my_union is %c\n", my_union.c);
    my_union.c = 'B';
    printf("The int in my_union is %d\n", my_union.num);
}
\end{lstlisting}

\section*{Activity 5 - Pointers}
Draw the memory diagram for the given program, is there an error? If there is, what is it? Otherwise
what is the output.
\begin{lstlisting}
#include <stdio.h>

  int main() {
      int a = 10;
      int x = 20;
      int *b = &a;
      int **c = &b;
      **c = 20;
      b = &x;
      *c = 30;
      *b = 30;
      return 0;
  }
\end{lstlisting}

\section*{Activity 6 - Enum Structs, Dynamic Memory}
Implement the function make cat using the code below

\begin{lstlisting}
/** Create declaration of the enum below **/







typedef struct cat_s {
    char *name;
    int age;
    enum Color color;
    void (*meow)();
} Cat;

void cat_speak() {
    printf("Meow\n");
}

Cat *make_cat(char *name, int age, enum Color color) {






}
\end{lstlisting}

\section*{Activity 8 - Processes}
Create a process that outputs the current working directory. You can use the command pwd to get
the current directory.
\end{document}
